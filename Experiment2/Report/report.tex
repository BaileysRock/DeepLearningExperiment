\documentclass[UTF8]{ctexart}
\usepackage{dirtree}
\usepackage{listings}
\usepackage{graphicx}
\usepackage{subfigure}
\usepackage{float}

\title{深度学习实验二} 
\author{1190200708 熊峰} 
\date{\today}
\begin{document} 
\maketitle 

\newpage
\tableofcontents
\newpage

\section{环境配置}
\subsection{硬件配置}
CPU : Intel(R) Xeon(R) Silver 4214 \par
GPU : TITAN RTX 24G \par
MEM : 128G RAM  \par
\subsection{软件配置}
OS : Ubuntu 20.04.1 LTS \par
PyTorch : Stable 1.11.0  CUDA 11.3 \par
IDE : PyCharm 2021.3.2 \par


\section{代码编写}

\subsection{数据读取}
将数据集下载好后,使用torchvision.transforms对数据预处理,将图片转化为Tensor,并正则化。\par
接下来对torch.utils.data.Dataset继承,并实现自己的数据集,在此步骤将数据的标签映射为数字,方便计算交叉熵损失函数。\par
在正式训练的时候使用data.DataLoader加载数据,并添加batch\_size。\par 
\subsection{搭建网络}
本实验实现了基于Pytorch的AlexNet、VGG16、InceptionV3、ResNet50、DenseNet121的模型。
其中AlexNet、VGG16使用PyTorch搭建完整模型,其余网络主要加载PyTorch的预训练模型,并进一步训练。\par
\subsubsection{AlexNet}
AlexNet网络结构如图所示。
\begin{figure}[H]
    \begin{center}
        \includegraphics[width=12cm]{\string"AlexNetModel".jpg}
    \caption{AlexNet Model}
    \label{fig:1}
    \end{center}
    \end{figure}
\par
原模型分为两部分在两个GPU中训练。
\begin{figure}[H]
    \begin{center}
        \includegraphics[width=12cm]{\string"AlexNetModel1".png}
    \caption{AlexNet Model}
    \label{fig:2}
    \end{center}
    \end{figure}
\par

本模型将两部分合并,网络主要由五层卷积层,三层池化层,和三个全连接层组成,原始输入为3*224*224的张量。\par
网络特点:在每个卷积层后加上ReLU激活函数,防止梯度消失问题,使整体收敛更快。使用Dropout避免过拟合。\par 
网络初始化方法主要采用何恺明提出的初始化方法。\par
网络结构(在卷积层后接ReLU,在线性层后接ReLU和DropOut):\par 
第一层:卷积层, kernel\_size = (11,11), stride = 4, in\_channel = 3, out\_channel = 96, 此时输出的size = (227-11+0)/4+1 = 55.\par 
第二层:池化层, kernel\_size = (3,3), stride = 2, 输出的size = (55-3)/2+1 = 27. \par 
第三层:卷积层, kernel\_size = (5,5), padding = 2, 输出的size = (27-5+4)/1+1 = 27. \par 
第四层:池化层, kernel\_size = (3,3), stride = 2, 输出的size = (27-3)/2+1 = 13. \par 
第五层:卷积层, kernel\_size = (3,3), padding = 1, 输出的size = (13-3+2)/1+1 = 13. \par 
第六层:卷积层, kernel\_size = (3,3), padding = 1, 输出的size = (13-3+2)/1+1 = 13. \par 
第七层:池化层, kernel\_size = (3,3), stride = 2, 输出的size = (13-3)/2+1 = 6. \par 
第八层:全连接层, in\_features = 9216, out\_features = 4096. \par 
第九层:全连接层, in\_features = 4096, out\_features = 4096. \par 
第十层:全连接层, in\_features = 4096, out\_features = 101. \par 


\subsubsection{VGG16}
\begin{figure}[H]
    \begin{center}
        \includegraphics[width=12cm]{\string"VGG16Model".jpg}
    \caption{VGG16 Model}
    \label{fig:3}
    \end{center}
    \end{figure}
\par
本部分主要实现了VGG16模型。\par 
第一部分经过两层卷积(在所有卷积操作后加上ReLU),此时输出为64*224*224.\par
第二部分经过一层池化层,再通过两层卷积层,此时输出为256*56*56.\par
第三部分经过一层池化层,再通过三层卷积层,此时输出为512*14*14.\par
第四部分经过一层池化层,此时输出为512*7*7.\par 
第五部分主要为全连接层,将上一部分输出映射为101维的向量。\par

\subsection{定义优化器}
优化器选择为Adam优化器。

\subsection{定义损失函数并训练}
本任务为多分类任务,因此将损失函数定义为交叉熵函数。

\section{实验验证}
\subsection{实验相关设置}
训练轮数 : 500(若五十轮以内验证集上没有优化,则结束训练)\par
训练设备 : GPU\par 
学习率 : 1e-5\par 
\subsection{训练过程}
实验使用tensorboard记录实验过程的数据。\par
\subsubsection{AlexNet}


\subsubsection{VGG16}

\subsubsection{ResNet50}

\subsubsection{InceptionV3}

\subsubsection{DenseNet121}


\subsection{实验结果}

\begin{center}
    \begin{tabular}{||c c c c c||}
    \hline
    模型 & Acc & Micro F1-Score & Macro F1-Score & Cross Entopy Loss\\ [0.5ex]
    \hline
    AlexNet & 69.48\% &  0.694841& 0.559827&0.741160\\
    VGG16 & \% &  & &\\
    InceptionV3 & 96.49\% & 0.964874 & 0.941263& 0.187301\\
    ResNet50 & 96.93\% & 0.969264 & 0.953177 & 0.137866\\
    DenseNet121 & 92.43\% &  0.924259 & 0.869438&0.338787\\
    \hline
   \end{tabular}
   \end{center}
\subsection{实验总结}


\end{document}
