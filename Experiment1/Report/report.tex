\documentclass[UTF8]{ctexart}
\usepackage{dirtree}
\usepackage{listings}
\usepackage{graphicx}
\usepackage{subfigure}
\usepackage{float}

\title{深度学习实验一} 
\author{1190200708 熊峰} 
\date{\today}
\begin{document} 
\maketitle 

\newpage
\tableofcontents
\newpage

\section{环境配置}
\subsection{硬件配置}
CPU : Intel(R) Xeon(R) Silver 4214 \par
GPU : TITAN RTX 24G \par
MEM : 128G RAM  \par
\subsection{软件配置}
OS : Ubuntu 20.04.1 LTS \par
PyTorch : Stable 1.11.0  CUDA 11.3 \par
IDE : PyCharm 2021.3.2 \par


\section{代码编写}

\subsection{数据读取}
使用torchvision.datasets.MNIST函数下载训练集和测试集,并使用transform对数据预处理。将数据正则化,并转化为张量。\par
在正式训练的时候使用data.DataLoader加载数据,并添加batch\_size。\par 
\subsection{搭建网络}
本实验实现了基于Pytorch的MLP及CNN的模型。\par
本实验中MLP网络主要分为三层。\par
第一层:将28*28维的向量映射为512维度的向量。\par
第二层:将512维的向量映射为512维度的向量。\par
第三层:将512维的向量映射为10维的向量。 \par

本实验除了MLP网络,还搭建了CNN网络,CNN网络主要分为三层。\par
第一层为卷积层,使用卷积核大小为(4,4),out\_channel设置为64,用于识别不同的模式。\par 
第二层为池化层,缩小参数矩阵。\par
第三层为全连接层,将数据映射为所需结果的维度。\par

\subsection{定义优化器}
优化器选择为Adam优化器。
\subsection{定义损失函数并训练}
本任务为多分类任务,因此将损失函数定义为交叉熵函数。

\section{实验验证}
\subsection{实验相关设置}
训练轮数 : 100(若十轮以内验证集上没有优化,则结束训练)\par
训练设备 : GPU\par 
学习率 : 1e-5\par 
\subsection{训练过程}
实验使用tensorboard记录实验过程的数据。\par
\subsubsection{MLP}

在训练12000轮后,实验结果趋于收敛。\par
\begin{figure}[H]
    % \flushleft
    \begin{center}
        \includegraphics[width=12cm]{\string"MlpAcc".png}
    \caption{准确率}
    \label{fig:1}
    \end{center}
    \end{figure}
\par

\begin{figure}[H]
    \begin{center}
        \includegraphics[width=12cm]{\string"MlpLoss".png}
    \caption{LOSS}
    \label{fig:2}
    \end{center}
    \end{figure}
\par


\subsubsection{CNN}

在训练8000轮后,实验结果趋于收敛。\par
\begin{figure}[H]
    % \flushleft
    \begin{center}
        \includegraphics[width=12cm]{\string"CnnAcc".png}
    \caption{准确率}
    \label{fig:3}
    \end{center}
    \end{figure}
\par

\begin{figure}[H]
    \begin{center}
        \includegraphics[width=12cm]{\string"CnnLoss".png}
    \caption{LOSS}
    \label{fig:4}
    \end{center}
    \end{figure}
\par






\subsection{实验结果}

实验得到的Confusion Matrix如下图所示:
\begin{figure}[H]
    \begin{center}
        \includegraphics[scale=0.4]{\string"confusionMLP".png}
    \caption{MLP Confusion Matrix}
    \label{fig:5}
    \end{center}
    \end{figure}
\par
\begin{figure}[H]
    \begin{center}
        \includegraphics[scale=0.4]{\string"confusionCNN".png}
    \caption{CNN Confusion Matrix}
    \label{fig:5}
    \end{center}
    \end{figure}
\par

实验结果如下表所示:

\begin{center}
    \begin{tabular}{||c c c||}
    \hline
    模型 & Acc & Micro F1\\ [0.5ex]
    \hline\hline\hline
    CNN & 97.52\% & 0.9751\\
    MLP & 95.22\% & 0.9483\\
    \hline
   \end{tabular}
   \end{center}


\subsection{实验总结}

通过动手实验MLP等模型,充实了理论的学习,收获颇丰。
\end{document}






